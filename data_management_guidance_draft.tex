\PassOptionsToPackage{unicode=true}{hyperref} % options for packages loaded elsewhere
\PassOptionsToPackage{hyphens}{url}
\PassOptionsToPackage{dvipsnames,svgnames*,x11names*}{xcolor}
%
\documentclass[a4paper,oneside]{report}
\usepackage{graphicx}

\usepackage{lmodern}
\usepackage{amssymb,amsmath}
\usepackage{ifxetex,ifluatex}
\usepackage{fixltx2e} % provides \textsubscript
\ifnum 0\ifxetex 1\fi\ifluatex 1\fi=0 % if pdftex
  \usepackage[T1]{fontenc}
  \usepackage[utf8]{inputenc}
  \usepackage{mathptmx} % provides euro and other symbols
\else % if luatex or xelatex
  \usepackage{unicode-math}
  \defaultfontfeatures{Ligatures=TeX,Scale=MatchLowercase}
\fi
% use upquote if available, for straight quotes in verbatim environments
\IfFileExists{upquote.sty}{\usepackage{upquote}}{}
% use microtype if available
\IfFileExists{microtype.sty}{%
\usepackage[]{microtype}
\UseMicrotypeSet[protrusion]{basicmath} % disable protrusion for tt fonts
}{}
\IfFileExists{parskip.sty}{%
\usepackage{parskip}
}{% else
\setlength{\parindent}{0pt}
\setlength{\parskip}{6pt plus 2pt minus 1pt}
}
\usepackage{xcolor}
\usepackage{hyperref}
\hypersetup{
            pdftitle={Data management in the field},
            pdfauthor={Jen Thomas, Swiss Polar Institute},
            colorlinks=true,
            linkcolor=blue,
            citecolor=Blue,
            urlcolor=Blue,
            breaklinks=true}
\urlstyle{same}  % don't use monospace font for urls
\usepackage[top=25mm,left=20mm,bottom=25mm,right=20mm,heightrounded]{geometry}
\setlength{\emergencystretch}{3em}  % prevent overfull lines
\providecommand{\tightlist}{%
  \setlength{\itemsep}{0pt}\setlength{\parskip}{0pt}}
\setcounter{secnumdepth}{0}

% set default figure placement to htbp
\makeatletter
\def\fps@figure{htbp}
\makeatother


\title{Data management in the field}
\author{Jen Thomas, Swiss Polar Institute}
\date{July 2020}

\usepackage{fancyhdr}
\pagestyle{fancy}
\fancyhead[LE,LO]{\includegraphics[scale=0.8]{spi_logo.png}}
\fancyfoot[CE,CO]{\thepage}
\renewcommand{\chaptermark}[1]{%
\markboth{\MakeUppercase{%
\chaptername}\ \thechapter.%
\ #1}{}}

\fancypagestyle{plain}{%
  \fancyhead[LE,LO]{\includegraphics[scale=0.7]{spi_logo.png}}
  \fancyfoot[CE,CO]{\thepage}
}

\fancypagestyle{titlepagestyle}
{
  \fancyhead[LE,LO]{\includegraphics[scale=1]{spi_logo.png} \\ EPFL ENT SPI \\ GR C2 505 \\ Station 2 \\ CH-1015 Lausanne, Switzerland}
  \fancyhead[CE,CO]{Phone: +41 21 693 76 74 \\ +41 21 693 10 74 \\ Web: www.swisspolar.ch}
  \fancyfoot[C]{\includegraphics[scale=0.25]{spi_funders_logos_png.png}}
  \renewcommand{\headrulewidth}{0pt}
}

\usepackage{xpatch}
\xapptocmd{\titlepage}{\thispagestyle{titlepagestyle}}{}{}




\begin{document}
\maketitle

{
\hypersetup{linkcolor=}
\setcounter{tocdepth}{1}
\tableofcontents
}
\hypertarget{introduction}{%
\chapter{Introduction}\label{introduction}}

\hypertarget{benefits-of-good-data-management}{%
\section{Benefits of good data
management}\label{benefits-of-good-data-management}}

Setting time aside for carefully managing data and samples from the
beginning of a project, will save a lot of time in the long run.
Investment of time in good preparation and management of data files,
software, documentation and associated files is an investment for the
future. Detailed documentation, organised files and well-kept backups
will make life a lot easier for your future self and others.

\hypertarget{what-this-guide-covers}{%
\section{What this guide covers}\label{what-this-guide-covers}}

In this guide we cover good research data management practice with a
focus on field work, although the general guidelines are applicable in
any circumstances. The first part of the guide is general and can be
applied throughout all stages of the research life cycle.

The second part is a guide for data management in the field and is
broken into three sections: planning data management for fieldwork, some
key points to remember whilst in the field and a brief set of
considerations for returning to your institution. We regularly refer
back to the key sections of the first part of the document and recommend
that the reader is familiar with this before reading the field guide.

\hypertarget{scope}{%
\subsection{Scope}\label{scope}}

Research data includes raw data directly recorded by hand or from an
instrument, processed data which are any files that have been modified,
documentation, metadata, supporting data files, code, plots and any
other files associated with your research.

This guide is applicable to all file types and ``kinds'' of data.

We also consider good practice of sample collection from a data
management point of view. This section has a particular focus on
numbering and collection of metadata about samples.

Finally, we consider field work, and more precisely polar field work
(although the same principles apply anywhere) in the broad sense of not
working at your institution. It could be at a field camp, on a
well-established base, on a ship or any other platform, the remote
operation of an instrument and wherever else you might be taken by your
research.

\hypertarget{further-development-of-this-guide}{%
\section{Further development of this
guide}\label{further-development-of-this-guide}}

We would greatly welcome feedback by way of comments, suggestions,
corrections or ideas for further development of this guide. Ongoing
development is done in a Github repository:
\url{https://github.com/Swiss-Polar-Institute/data-management-guidance}.
Feel free to get in touch using
\href{https://github.com/Swiss-Polar-Institute/data-management-guidance/issues}{Github
issues} or email \url{jenny.thomas@epfl.ch}.

\hypertarget{data-management---good-practice}{%
\part{Data management - good
practice}\label{data-management---good-practice}}

\hypertarget{storing-data}{%
\chapter{Storing data}\label{storing-data}}

Storing your data in an organised and secure manner, will save you a lot
of time and hassle in the long run. It is worth investing time and
effort in ensuring a coherent
\protect\hyperlink{directory-structure}{directory structure},
understandable \protect\hyperlink{file-naming}{file names}, secure data
storage and carefully thought-out
\protect\hyperlink{data-backup}{backups}.

In this section we focus on where to store your data. Suggestions are
\emph{not} ordered by preference.

\hypertarget{what-to-consider-when-deciding-where-to-store-your-data}{%
\section{What to consider when deciding where to store your
data}\label{what-to-consider-when-deciding-where-to-store-your-data}}

Thinking carefully about what you need to be able to do with your data
will help you to select where is best to store your data.

\begin{itemize}
\tightlist
\item
  What do you want to do with the data?
\item
  Do other people need access to your data?
\item
  What is the volume of your data?
\item
  How long do you want to use the particular storage for?
\item
  What budget do you have for data storage?
\item
  How much does the data storage cost?
\item
  Are your data sensitive?
\end{itemize}

\hypertarget{types-of-storage-media}{%
\section{Types of storage media}\label{types-of-storage-media}}

Remember that for \protect\hyperlink{data-backup}{backups} of data, it
is important to have them in multiple places and on different types of
media. Different types of media offer solutions to different problems
and therefore you will need to consider using more than one.

\hypertarget{networked-data-storage}{%
\subsection{Networked data storage}\label{networked-data-storage}}

Networked data storage, often known as an institution file server, is
often provided as standard by institutions.

\begin{itemize}
\tightlist
\item
  This is often a handy place to store your data for easy access whilst
  working at your institution.
\item
  Always familiarise yourself with the backup schedule.
\item
  Check if recovery of previous versions of files is necessary (it might
  only be possible if a hard drive is corrupt for example, rather than
  if you overwrite a file by mistake). Find out if you are able to
  recover previous versions of files yourself, or if you need the help
  of an administrator.
\item
  Check access rights: would you have a personal area of data storage,
  or would it be shared amongst your lab or a wider group? Think if you
  need the data to have restricted access, particularly if it is
  sensitive (protected species) or contains personal data. Access should
  also be limited to avoid changing or deleting files accidentally. Make
  sure that primary raw data are read-only for yourself and others.
\item
  It is worth checking if it is easy to allow external collaborators to
  access your data, although this is more often done through other
  means.
\item
  Check what limits there are on the volume of storage: if you are
  collecting large volumes of data, you may need to budget for asking
  for more data storage.
\item
  Consider that you may need to access the data from off-site from time
  to time and find out early on how to do this. Make sure it works,
  there is nothing worse than trying to access data (or other file) that
  you need in a hurry and not have what you need set up properly.
\end{itemize}

\hypertarget{personal-computers}{%
\subsection{Personal computers}\label{personal-computers}}

If using a personal computer to work on data, ensure that you have a
master copy backed up elsewhere, that you work on a \emph{copy} of the
\protect\hyperlink{working-on-your-data}{raw data}, and that any work
you do is backed up on a regular basis.

Failure of the hard disk, theft or damage to a personal computer or
laptop itself could mean loss of all the data and associated files.

\hypertarget{portable-media}{%
\subsection{Portable media}\label{portable-media}}

Portable hard drives are commonly used for
\protect\hyperlink{data-backup}{backing up} data in the field but their
use-case should be carefully considered. Some portable media types are
not useful for long-term storage because they quickly degrade or become
obsolete.

\begin{itemize}
\tightlist
\item
  Buy reputable makes of portable hard drives. If you are going to use a
  lot of them (for example, during a particular field season), by at
  least two different makes to avoid buying a ``bad'' batch.
\item
  Consider buying several medium-sized hard drives rather than one large
  one. If one fails, at least you do not lose everything.
\item
  Take care of hard drives: remember that they are susceptible to
  physical damage and depending on how many times you write to them, may
  only last a few years.
\item
  Pen drives are easy to lose and shouldn't be considered as reliable
  for data backup.
\item
  Consider how long it takes to back up your data and bear this in mind.
  If using an older device, it may have USB-2 only which is considerably
  slower than USB-3.
\item
  If a hard drive is lost, it can be easily read by anyone. The hard
  drive should be encrypted if it holds any personal or sensitive data.
\item
  Regularly check any data that are held on portable media to ensure it
  can still be read. Always have other backups.
\end{itemize}

\hypertarget{cloud-storage}{%
\subsection{Cloud storage}\label{cloud-storage}}

Cloud storage is becoming the norm in many cases, particularly where the
data volume is getting into hundreds of GBs. There are many types of
cloud storage that you can set up yourself or buy as a managed service,
but there are a few considerations to take into account. Many
institutions now also offer this either as part of their standard
storage or as an extra service, particularly for those dealing with
larger volumes of data. Research data management staff in your
institution may also be able to offer recommendations.

\begin{itemize}
\tightlist
\item
  If you are storing personal data, it is essential to understand the
  location of the physical servers and ensure this complies with
  regulations such as GDPR or of your institution.
\item
  Access rights are also important. Consider who might have access to
  the data, if you use read-only or write access rights.
\item
  Check the privacy policy carefully.
\item
  Some cloud storage providers charge not only for the data storage, but
  also for the number of files, copying data to and from the storage, as
  well as listing files (known as objects in cloud storage). It is
  important to think about how this could impact on your costs.
\item
  Some providers provide different levels of storage: consider if you
  want to have immediate access to your data or if you are happy to have
  it in ``cold storage'' where it may take a while to access it - this
  is often a low-cost option and good for long-term backups of data that
  you are not actively working on.
\item
  For managed systems, make sure that backups are done regularly.
\item
  For unmanaged systems, check how you will be able to copy and access
  data. Many systems require the use of command-line tools such as
  \texttt{rclone}.
\item
  Ensure you understand how to give others access to the data.
\item
  Speed of access to data (particularly when more than a few GBs) can be
  limited by your bandwidth - consider how you will access large volumes
  of data carefully.
\item
  Always check that you are able to move your data to another provider
  at a reasonable cost and in a reasonable time manner. This could
  happen if there are changes to the services that are provided, they go
  out of service or they no longer meet the requirements you need. Some
  providers may use proprietary formats which might make moving the data
  very hard.
\item
  Ensure the subscription is always maintained (if paid for) otherwise
  you may find your data are deleted.
\end{itemize}

Wikipedia has a very handy
\href{https://en.wikipedia.org/wiki/Comparison_of_online_backup_services}{comparison
of online backup options}. These would normally be used for medium to
long-term storage or backups.

Platforms such as \href{https://zenodo.org}{Zenodo} are provided for
publication of data and other digital resources, but when datasets have
been completed (either in a raw or finalised state) it can be worth
thinking about this option.

Object storage has various idiosyncrasies in terms of differences to
file systems that are useful to be aware of. For example, files are
known as ``objects'', S3 prefixes are not directories (Chan, 2020a) and
S3 keys are not file paths (Chan, 2020b).

\hypertarget{file-organisation}{%
\section{File organisation}\label{file-organisation}}

It is important to strike a balance between the number of files you
produce and their size. Consider the size of data files and the number
of files you store in a directory.

\hypertarget{file-size}{%
\subsection{File size}\label{file-size}}

Consider file size carefully so they are easy to work with for yourself,
future users and applications. Many small files will take longer to copy
and be harder to work with than a single file of the same total size.
Equally, avoid creating files that are more than 1 GB because in some
cases they can be difficult to read into memory. Copying lots of smaller
files to \protect\hyperlink{cloud-storage}{cloud storage} can also
increase the cost which maybe a factor.

\hypertarget{number-of-files-in-a-directory}{%
\subsection{Number of files in a
directory}\label{number-of-files-in-a-directory}}

TODO

\hypertarget{references}{%
\section{References}\label{references}}

Chan, A. (2020). alexwlchan. \emph{S3 keys are not file paths}.
Retrieved from
\url{https://alexwlchan.net/2020/08/s3-keys-are-not-file-paths/}
{[}Accessed on 15 September 2020{]}.

Chan, A. (2020). alexwlchan. \emph{S3 prefixes are not directories}.
Retrieved from
\url{https://alexwlchan.net/2020/08/s3-prefixes-are-not-directories/}
{[}Accessed on 15 September 2020{]}.

Craig-Wood. N. Rclone. (2014-2020). \url{https://rclone.org}

The University of Edinburgh. Storage \& security. \emph{MANTRA Research
Data Management Training}. Retrieved from
\url{https://mantra.edina.ac.uk/storageandsecurity} {[}Accessed on 28
July 2020{]}.

\hypertarget{directory-structure}{%
\chapter{Directory structure}\label{directory-structure}}

An organised directory structure will work wonders when it comes to
looking for files you haven't been working on for a while. Think about a
hierarchical structure that could be repeated across all of your
projects and one that can be easily navigated by others.

\hypertarget{handy-tips}{%
\section{Handy tips}\label{handy-tips}}

\begin{itemize}
\tightlist
\item
  Don't use spaces or special characters in directory names as these
  cannot always be handled easily. Instead, consider splitting the
  sections using underscores ( \_ ) or dashes ( - ) and writing words
  within sections using CamelCaps (starting each word with a capital
  letter). This can help with readability.
\item
  Keep names short but meaningful: some file systems have limits on the
  number of characters in a full directory path so if you have several
  sub-directories this can become a problem.
\item
  Use acronyms sparingly and if necessary, use only those that are
  well-known.
\item
  Your directories will be most likely be listed in alphabetical order.
  Prefixes such as numbers or letters to order your directories are not
  helpful if they do not mean anything. Dates could be used if
  appropriate but the most important thing is to name them carefully so
  they are meaningful and provide good documentation in a
  \protect\hyperlink{readmetxt}{readme}.
\item
  If you are going to store thousands or millions of files, consider a
  more hierarchical structure such as
  YYYY/MM/DD/YYYYMMDD\_data\_file\_name. Listing many files within one
  directory can take a long time, making them harder to work with.
\item
  Working in a separate directory to your raw or finalised files is good
  practice to avoid accidentally modifying them. Consider creating a
  work-in-progress (``wip'') directory which could have different
  permissions and a different \protect\hyperlink{data-backing}{backup
  schedule}.
\item
  Include a \protect\hyperlink{readmetxt}{readme} file to describe what
  the directory contains.
\item
  Think carefully about
  \protect\hyperlink{number-of-files-in-a-directory}{how many files are
  stored in a directory} to ensure they are easy to work with.
\end{itemize}

A good example can be seen here.

\begin{verbatim}
ProjectName/
|-- Documentation
|-- Plots
|-- ProcessedData
|-- RawData
|   |-- FieldSiteA
|   |   |-- 2018
|   |   |-- 2019
|   |   |-- 2020
|   |-- FieldSiteB
|   |   |-- 2018
|   |   |-- 2019
|   |   |-- 2020
|   |-- FieldSiteC
|       |-- 2018
|       |-- 2019
|       |-- 2020
|-- readme.txt
|-- wip
\end{verbatim}

\hypertarget{readme.txt}{%
\section{readme.txt}\label{readme.txt}}

\textbf{Always} include a readme file within a directory to describe its
contents. This will help anyone coming to look at the files in the
future, and will help jog your memory as well.

The Gurdon Institute (Downie, 2019) provides a very useful list of what
to include here, summarised as follows:

\begin{itemize}
\tightlist
\item
  summarise what is in the directory
\item
  use keywords for the project, data type or parameters so that they can
  be searched in the future
\item
  include the name of the person(s) who created the directory and their
  contact details
\item
  describe any changes made to the directory and when
\item
  make sure the file is written in text format (.txt) so that it can
  easily be read in the future.
\end{itemize}

If you would like to write with formatting, Markdown is a useful
convention that can be used. \texttt{Pandoc} is a very useful tool to
then convert Markdown into different file formats such as HTML, PDF etc.
as you so wish. Remember to save the readme in text format though so
that it can be easily opened in years to come.

\hypertarget{references-1}{%
\section{References}\label{references-1}}

Borer, E. T., Seabloom, E. W., Jones, M. B. and Schildhauer, M. (2009).
Some Simple Guidelines for Effective Data Management, Ecological Society
of America, 90(2), pp.~205--214. doi:
\href{https://doi.org/10.1890/0012-9623-90.2.205}{10.1890/0012-9623-90.2.205}.

Downie, A. (2019) Bite-sized RDM \#5 - the readme file. \emph{IT and
Research Data Management at the Gurdon Institute}. Retrieved from
\href{https://gurdoncomputing.blog/2019/12/02/bite-sized-research-data-management-5-the-readme-file/}{https://gurdoncomputing.blog/2019/12/02/bite-sized-research-data-management-5-the-readme-file}
{[}Accessed 27th February 2020{]}.

\hypertarget{file-naming}{%
\chapter{File naming}\label{file-naming}}

Having an organised directory structure and standard procedure for file
naming can be essential in helping to locate files in the future. It
also avoids a lot of confusion for anyone using the files.

\hypertarget{handy-tips-1}{%
\section{Handy tips}\label{handy-tips-1}}

Some file naming tips (Borer et al., 2009):

\begin{itemize}
\tightlist
\item
  Use a filename which is concise and accurately reflects what is
  contained within the file. Splitting it into separate parts, such as
  project, title, year or location of collection, year of collection,
  data type, version number and the file type can help to have a
  hierarchical name and standard naming procedure. Using keywords can
  help you find the file at a later date.
\item
  Stick to letters and numbers: special characters (non-ASCII
  characters) are unfortunately not well-supported by some software and
  can cause problems.
\item
  Underscores ( \_ ) and dashes ( - ) are conventionally used for
  separation of different parts of a filename. CamelCaps (starting each
  word with a capital letter) can be used to separate words within
  filename sections. Avoid using spaces because this can cause some
  problems with different systems.
\item
  Versioning can be done using the date in the format YYYYMMDD. Placing
  the date at the start of the filename can be useful. If more
  granularity is useful, then version numbering such as v01\_01, v01\_02
  can be used as well. Placing this at the end of the filename is useful
  although if you are getting into detailed versioning such as this, you
  should consider using a
  \protect\hyperlink{working-on-your-data}{versioning tool}.
\item
  Be consistent!
\end{itemize}

\hypertarget{examples}{%
\section{Examples}\label{examples}}

\begin{verbatim}
ace_meteorology_data_20170234-120000.csv
\end{verbatim}

\begin{itemize}
\tightlist
\item
  ace is the overarching project
\item
  meteorology is the sub-project
\item
  data signified that this is data rather than documentation
\item
  20170234-120000 is the first timestamp of data in the file
\item
  .csv is the file type (comma-separated values)
\end{itemize}

\begin{verbatim}
ace_meteorology_ProcessedWindData_201701.csv
\end{verbatim}

\begin{itemize}
\tightlist
\item
  ace is the overarching project
\item
  meteorology is the sub-project
\item
  ProcessedWindData is information about the data contained in the file
\item
  201701 is the subset of data in the file (only data from January 2017)
\item
  .csv is the file type (comma-separated values)
\end{itemize}

\hypertarget{references-2}{%
\section{References}\label{references-2}}

Borer, E. T., Seabloom, E. W., Jones, M. B. and Schildhauer, M. (2009).
Some Simple Guidelines for Effective Data Management, Ecological Society
of America, 90(2), pp.~205--214. doi:
\href{https://doi.org/10.1890/0012-9623-90.2.205}{10.1890/0012-9623-90.2.205}.

\hypertarget{data-file-formats}{%
\chapter{Data file formats}\label{data-file-formats}}

The format of the files you use to store data and documentation will
directly impact how this information can be used by yourself and others
in the future (Stanford Library, 2020)

Using open data file formats helps to ensure the longevity of datasets
(Borer et al., 2009). Open file formats are well-documented and easy to
read by a variety of software and are more future-proof. Using file
formats that are closed and specific to a certain piece of software
(proprietary), have a higher probability of becoming unreadable in the
future. As software versions change, they are not always backwards
compatible, meaning that a file produced ten years ago may no longer be
readable. Trends also change and now-common software applications may
not be widely used in the future.

\hypertarget{handy-tips-2}{%
\section{Handy tips}\label{handy-tips-2}}

File formats should be (as described by MIT information on storing your
data, CC BY-NC):

\begin{itemize}
\tightlist
\item
  open and documented
\item
  commonly used by the research community
\item
  unencrypted
\item
  uncompressed
\item
  use standard representation (ASCII, Unicode)
\end{itemize}

For tabular data:

\begin{itemize}
\tightlist
\item
  CSV files are an easy solution if you have minimal metadata contained
  within the file. Otherwise HDF5 is a good option.
\item
  if converting data from a proprietary format to an open format, ensure
  that no data or meaningful information is lost. If this is going to be
  the case, it is worth considering keeping both copies and thoroughly
  document the proprietary software needed to create and read the
  proprietary files (name, version, operating system; DataONE).
\end{itemize}

For documentation:

\begin{itemize}
\tightlist
\item
  plain text (TXT) files are simple and easily read.
\item
  for structured information, you could consider ODF, LaTeX or Markdown
  for example.
\item
  tools such as \href{https://frictionlessdata.io/}{Frictionless Data}
  are very useful for providing machine-readable metadata about your
  datasets.
\end{itemize}

For other types of data (such as media or geospatial, for example)
consult EPFL's
\href{https://www.epfl.ch/campus/library/wp-content/uploads/2019/09/EPFL_Library_RDM_FastGuide_All.pdf\#page=4}{Research
Data Management Fast Guide}.

\hypertarget{useful-links}{%
\section{Useful links}\label{useful-links}}

Page 4 of EPFL's
\href{https://www.epfl.ch/campus/library/wp-content/uploads/2019/09/EPFL_Library_RDM_FastGuide_All.pdf\#page=4}{Research
Data Management Fast Guide} has a useful summary of appropriate file
formats to use for different types of data.

Frictionless Data
\href{https://specs.frictionlessdata.io/table-schema/}{table schema} and
\href{https://specs.frictionlessdata.io/data-package/}{data package}
schemas are very useful for describing tabular data and datasets
respectively.

\hypertarget{references-3}{%
\section{References}\label{references-3}}

Blumer, E.N., Chaptinel, J.J., Masson, A., Reichler, F., Samath, S.,
Varrato, F. and Milfort, F. (2019). EPFL Library Research Data
Management Fastguides. Zenodo. doi:
\href{https://doi.org/10.5281/zenodo.3327830}{10.5281/zenodo.3327830}

DataONE. Document and store data using stable file formats.
\emph{DataONE}. Retrieved from
\url{https://old.dataone.org/best-practices/document-and-store-data-using-stable-file-formats}
{[}Accessed on 24 July 2020{]}

MIT. Formats. \emph{Data management}. Retrieved from
\href{https://libraries.mit.edu/data-management/store/formats/}{https://libraries.mit.edu/data-management/store/formats}
{[}Accessed on 24 July 2020{]}

Stanford Library. Best practices for file formats. \emph{Stanford
Libraries}. Retrieved from
\url{https://library.stanford.edu/research/data-management-services/data-best-practices/best-practices-file-formats}
{[}Accessed on 24 July 2020{]}

\hypertarget{data-backup}{%
\chapter{Data backup}\label{data-backup}}

Ensuring you have several reliable copies of your data avoids data loss
and gives you peace of mind. Think carefully about the type of
\protect\hyperlink{storing-data}{data storage} you use for your backups.

It is important to remember that if you are working in the field, your
backup set-up and schedule will differ from your backups that are done
of other data and when working back at your institution. This is covered
more detail in \protect\hyperlink{before-you-go}{planning} and
\protect\hyperlink{in-the-field}{working in the field} sections of the
guide.

\hypertarget{handy-tips-3}{%
\section{Handy tips}\label{handy-tips-3}}

Make sure:

\begin{itemize}
\tightlist
\item
  you have at least two, preferably three or more
  \protect\hyperlink{how-much-and-how-often}{copies} of your data;
\item
  data are backed up on at least two different
  \protect\hyperlink{storing-data}{types of media}, particularly for
  preservation purposes, such as institution storage, cloud storage,
  external hard drives;
\item
  that as far as possible, backups are automated. This avoids potential
  mistakes, minimises the chances of data loss, makes it much easier to
  do (it is less of a chore) and ensures the backups are always done in
  the same way;
\item
  data are backed up on a
  \protect\hyperlink{how-much-and-how-often}{regular basis}, but
  particularly during collection and after making any changes;
\item
  that backed up versions of your data are identical to the primary
  copy. Whilst
  \protect\hyperlink{checking-and-restoring-backups}{checking} that
  files have been copied, even if they are listed in the secondary
  location, using checksums will confirm they have been copied
  correctly;
\item
  that you can easily
  \protect\hyperlink{checking-and-restoring-backups}{restore} your
  backups;
\item
  decide on a \protect\hyperlink{storing-data}{directory structure} and
  \protect\hyperlink{file-naming}{file naming} convention for your data
  and stick to it. Making changes to these (unless absolutely necessary)
  can create problems with backups because it is easy to lose track of
  what has been copied and what hasn't, which is the latest version and
  so on;
\item
  if you will be backing up tens of GB or more of data, bear in mind how
  long a backup will take and bear that in mind when deciding how you
  will arrange your backup schedule.
\end{itemize}

\hypertarget{creating-a-backup-schedule}{%
\section{Creating a backup schedule}\label{creating-a-backup-schedule}}

\hypertarget{how-much-and-how-often}{%
\subsection{How much and how often}\label{how-much-and-how-often}}

Think carefully about how often you want to backup your data and if you
will do full backups each time, or only partial backups for files that
have changed. Don't forget that your documentation, code, plots and
other associated files should also be backed up alongside the data.
Automating your backups will make everything much simpler and help to
avoid mistakes.

Arguably it is much simpler to do a full backup of your files each time
and retain these for a certain period of time. If you are not working on
the data or associated files any longer, then as long as the backups are
secure and regularly checked, this could be done every month, for
example.

If however you are working on your files on a daily basis, having daily
backups can save a lot of time if you notice an issue. In this case, you
may consider to do daily backups of only a subset of the files.

\hypertarget{size-of-backup}{%
\subsection{Size of backup}\label{size-of-backup}}

It is important to consider how much space each backup will take and
therefore how much total space you need available for all of your
backups. Data volume and also the number of files will both affect how
long it takes to do a backup.

\hypertarget{retention-of-backups}{%
\subsection{Retention of backups}\label{retention-of-backups}}

If using managed data storage such as from your institution, be sure to
understand how often backups are done and for how long these are
retained. It is also important to be able to
\protect\hyperlink{file-restoration}{restore your files easily}.

For your regular backups, depending on how often you do your backups,
you might decide to have a backup cycle where you do one per month and
retain them for six months, for example.

If data are being stored for preservation purposes then you need to take
into account how the data have been collected, if they have been
published openly anywhere and their importance for the future. This
should be planned in a data management plan.

Consider also that you might not realise it now, but these data could be
part of a long-term study in the future. For example, you might start
collecting a set of time series data for a one-off project. If this
project gets another set of funding for a second year in a row, you will
then have amassed two years' worth of data. Who knows if you might end
up then setting up an observing programme at the same site with
automated monitoring for the next ten years. Being able to read and
access the initial ``one-off'' dataset years into the future can provide
an opportunity for a publication that you might not originally have
envisaged right back at the beginning.

\hypertarget{documenting-backups}{%
\section{Documenting backups}\label{documenting-backups}}

It is important to document where the files have been backed up, when,
how often, as well as how they can be accessed and restored if
necessary. Think of this as a backup for your future self - it might be
a whole field season before you look back at how you organised the
backups, so after a long time away, it is useful to pick up a short
README file with this information and be able to restore backed-up data
straight away.

\hypertarget{checking-and-restoring-backups}{%
\section{Checking and restoring
backups}\label{checking-and-restoring-backups}}

\hypertarget{checksums}{%
\subsection{Checksums}\label{checksums}}

It is important to check the backup when it has been completed to ensure
it contains the files you expect. Checksums are a unique identifier of a
file: if it's content changes in any way, then it's checksum also
changes. Comparing checksums of your original files and the backup is a
very handy way to ensure that the backup contains the files you expect.

\href{https://en.wikipedia.org/wiki/Md5sum}{md5sum} and
\href{https://en.wikipedia.org/wiki/Sha1sum}{sha1sum} are two examples
of computer programs that compute checksums of files and would be
suitable for this purpose.

\hypertarget{file-restoration}{%
\subsection{File restoration}\label{file-restoration}}

Storage media can become obsolete, file permissions and access can be
changed accidentally, and subscriptions to services are sometimes not
renewed. There can be many possible ways in which file backups are lost.
Regularly checking that you can still read and restore the data is
important to ensure that there are no problems.

Choosing a simple, well-understood, transparent and multi-platform tool
will often make file restoration much simpler.

If any of your data were produced using proprietary software it is
particularly important to ensure that you can still read them on a
regular basis. You may need a specific piece of software that requires a
license (do you still have the license?) or even a specific version of
software that might become outdated. Consider outputting data into an
\protect\hyperlink{data-file-formats}{open, documented format} such as
txt or csv - be aware that in this process you may lose some information
or data, so it is always good practice to keep both sets of files.

\hypertarget{backup-tools}{%
\section{Backup tools}\label{backup-tools}}

Many tools are available to help create backups. It is worth spending
time finding one that meets your needs taking into account future, as
well as current, needs. Of course it is essential to fully understand
how the data are being saved and how they can be recovered.

Your institution may have tools or subscriptions to tools that can help.
Data managers or librarians within your institution may also be able to
offer recommendations.

Using a multi-platform tool (usable by Windows, Mac and Linux users)
offers higher resilience, ensuring more possibilities for accessing the
data in the future and allow access to other team members.

\hypertarget{references-4}{%
\section{References}\label{references-4}}

The University of Edinburgh. MANTRA Research Data Management Training.
Storage \& security. Retrieved from
https://mantra.edina.ac.uk/storageandsecurity/ {[}Accessed on 28 July
2020{]}.

Wikipedia. Checksum. \emph{Wikipedia The Free Encyclopedia}. Retrieved
from \url{https://en.wikipedia.org/wiki/Checksum} {[}Accessed on 28 July
2020{]}.

Wikipedia. md5sum. \emph{Wikipedia The Free Encyclopedia}. Retrieved
from \url{https://en.wikipedia.org/wiki/Md5sum} {[}Accessed on 15
September 2020{]}.

Wikipedia. sha1sum. \emph{Wikipedia The Free Encyclopedia}. Retrieved
from \url{https://en.wikipedia.org/wiki/Sha1sum} {[}Accessed on 15
September 2020{]}.

\hypertarget{working-on-your-data}{%
\chapter{Working on your data}\label{working-on-your-data}}

Here we focus on ensuring data security (not losing raw or processed
data), integrity (not changing raw data files) and capturing the
provenance of datasets.

Careful recording of how a dataset has been produced is important
firstly so that you or someone else can understand what has been done.
Documenting steps as you go along is a useful reminder for when it comes
to writing up your work for publication or for when you go back to what
you were doing after a long field season away. Making this information
clearly available alongside the dataset to someone else such as a future
user of your data, would allow them to reproduce your work.
Reproducibility is becoming ever more important to ensure scientific
validity (Peng, 2012).

\hypertarget{raw-data}{%
\section{Raw data}\label{raw-data}}

Your primary copy of the data, known as ``raw'' data, is that which
comes directly from the instrument, sample analysis, or primary
observation. This raw data should be saved as it is, backed up and aside
from being copied elsewhere, \textbf{should never be modified}. The
primary copy of the raw data should have read-only access so it can
never be altered inadvertently. Never reorganise the primary copy of raw
data files.

\hypertarget{work-on-a-copy-of-your-raw-data}{%
\section{Work on a copy of your raw
data}\label{work-on-a-copy-of-your-raw-data}}

In preparation for working on the data, performing quality checking,
applying calibrations, or indeed making any change whatsoever to the
data file, make a copy of the raw data and ensure you work on the copy.
Ensuring you maintain a ``pristine'' version of the raw data is
imperative: throughout the analysis process it is possible to discover
inconsistencies and errors that may require you to go back to the raw
data files in order to find their origins. Any changes that have
subsequently been made may mean that it is then impossible to track down
the origins of these problems.

\hypertarget{versions-of-files}{%
\section{Versions of files}\label{versions-of-files}}

Using scripted languages to do any file manipulation is the ideal way to
record what you have done to data files and demonstrate how a
preliminary dataset has been created. Steps can easily get lost,
forgotten or critical errors introduced without you realising (eg.
Ziemann et al., 2016), if using spreadsheets or making direct changes.
If you do use such methods, maintaining full documentation about every
step of the processing and outputting different versions of the dataset
at crucial stages is a way to ensure you can demonstrate the provenance
of a dataset.

When working on processing, quality checking and making other changes to
produce the preliminary dataset, creating different versions of the
files after each step rather than overwriting files, is a valuable way
of ensuring you can follow all steps that have been undertaken and allow
you to go back to a certain place if you find errors or want to change
your methods.

File versions can be named using the date (in the format YYYYMMDD) or
version numbers, such as \texttt{v01\_01}, \texttt{v01\_02}. Including
the date in this format, or version numbers that have leading zeros
ensure that files are listed in order when viewing them.

\protect\hyperlink{backing-up-data}{Backups} of any edited files should
be done on a regular basis.

\hypertarget{versioning-tools}{%
\subsection{Versioning tools}\label{versioning-tools}}

\begin{itemize}
\tightlist
\item
  Git
\item
  Git-LFS
\end{itemize}

\href{https://github.com/}{Github} is commonly used as a platform for
software versioning and is becoming more widely used for versioning of
other non-binary files as well.

\hypertarget{recording-the-provenance-of-your-data}{%
\section{Recording the provenance of your
data}\label{recording-the-provenance-of-your-data}}

It is natural to keep notes of what you have done to data so that when
writing up publications, you are able to explain your methods. It is
often impossible to repeat data collection, which in polar research is
also often extremely expensive. This ``original'' data is considered as
\protect\hyperlink{raw-data}{raw} and should be kept as such. But any
\protect\hyperlink{work-on-a-copy-of-your-raw-data}{work that is done on
a copy of this raw data} should be carefully recorded.

Journals are more and more asking for all supporting documentation,
code, data and information about how plots and figures were generated,
so organising to capture the full provenance of your data and research
paper will save you time when you come to publish.

\begin{itemize}
\tightlist
\item
  Wherever possible, use scripts to do any manipulation of data, for
  applying algorithms, quality-checking and any other processes that
  work towards your final, output dataset. Scripts can be modified
  easily if you spot a mistake and then re-run, rather than having to
  run through all of your manual steps again.
\item
  Backup and keep different
  \protect\hyperlink{versioning-tools}{versions} of your code so that
  you can see where possible errors are introduced into your data or
  processing.
\item
  Make sure you also document what you are doing: clearly state
  references to algorithms, which software you are using (including the
  version) as well as which decisions you have taken and why.
\item
  Ideally, you should be able to record the set-up of the computing
  environment that you have used to run the scripts: include details of
  the operating system, package names and versions.
\item
  Always keep data and code that produces plots: this could not only
  save you time if you spot a mistake you want to correct but it could
  be required to publish it in a journal.
\end{itemize}

Refer to the \protect\hyperlink{metadata}{metadata} section for full
details of what information should be captured to properly describe the
provenance of your data and samples.

\hypertarget{capturing-data-provenance-tools}{%
\subsection{Capturing data provenance
tools}\label{capturing-data-provenance-tools}}

\protect\hyperlink{versioning-tools}{Versioning tools} are very handy
for making incremental improvements to code and even versions of data in
the case of Git-LFS.

Some tools also exist which allow you to record exactly what has
happened to your dataset, recording its provenance.

\begin{itemize}
\tightlist
\item
  RENKU \url{https://datascience.ch/renku/}
\item
  Whole Tale \url{https://wholetale.org/}
\end{itemize}

\hypertarget{references-5}{%
\section{References}\label{references-5}}

Peng R. D. (2011). Reproducible research in computational science.
Science (New York, N.Y.), 334(6060), 1226--1227. doi:
\href{https://doi.org/10.1126/science.1213847}{10.1126/science.1213847}

Ziemann, M., Eren, Y., \& El-Osta, A. (2016). Gene name errors are
widespread in the scientific literature. Genome Biology, 17(1), 177.
doi:
\href{https://doi.org/10.1186/s13059-016-1044-7}{10.1186/s13059-016-1044-7}

\hypertarget{collecting-samples}{%
\chapter{Collecting samples}\label{collecting-samples}}

Obtaining accurate results from your experiment when analysing samples,
depends on ``proper collection, processing and handling of samples''
(Smith et al., 2015), planning for which begins before sample
collection.

Our main aims when considering good practice for collection of samples
are:

\begin{itemize}
\tightlist
\item
  accurate labelling of samples;
\item
  quickly identifying or finding a sample;
\item
  matching it to any documentation or metadata about how it was
  collected.
\end{itemize}

Here we focus on numbering samples and recording information about their
collection.

\hypertarget{numbering-samples}{%
\section{Numbering samples}\label{numbering-samples}}

Carefully planned sample numbers will make it much easier to meet our
aims above. Keep numbers as simple as possible to avoid mistakes and
confusion.

\hypertarget{requirements-from-other-partners}{%
\subsection{Requirements from other
partners}\label{requirements-from-other-partners}}

If your samples belong to a larger project, make yourself aware of any
requirements that should be met. All projects will need to ensure that
sample numbers are unique, therefore you might be required to add a
specific prefix or suffix to identify your samples from those of another
part of the project.

The laboratory or organisation where the samples are going to be stored
and / or analysed may also have their own requirements for numbering.

Making yourself aware of these requirements in advance will make sure
you are not caught out having to make amendments not only to the numbers
written on the samples themselves, but your documentation and metadata
as well.

\hypertarget{unique-identification-number-for-each-sample}{%
\subsection{Unique identification number for each
sample}\label{unique-identification-number-for-each-sample}}

Each sample should have a unique identification number written on it -
don't just write a short version of a longer sample number that could
easily be confused if more than one sample from your field campaign (or
the wider project) may end up with the same number.

This unique identification number will then ``link'' that physical
sample to the information that is recorded about its collection
(\protect\hyperlink{recording-information-about-sample-collection-metadata}{metadata}).

\hypertarget{duplicate-samples}{%
\subsection{Duplicate samples}\label{duplicate-samples}}

In some experiments, more than one sample will be taken at the same time
and under the same conditions in order to corroborate results. These
samples may be known as ``duplicates'' (or ``triplicates'' in the case
of three, and so on). It is important to be able to distinguish between
these samples.

\hypertarget{further-considerations}{%
\subsection{Further considerations}\label{further-considerations}}

\begin{itemize}
\tightlist
\item
  The space that you have on your sample container.
\item
  You might be able to label containers before sample collection - be
  very aware of mistakes that can be made if for some reason you need to
  put a sample in the ``wrong'' or a ``wrongly labelled'' container.
\item
  Some groups may use a bar coding system to record and label samples.
  There are various costs associated with setting up and maintaining
  such a system, but it may be worth consideration and the investment if
  you are going to be collecting thousands of samples (Copp et al.,
  2014).
\end{itemize}

\hypertarget{labelling}{%
\section{Labelling}\label{labelling}}

Label containers directly using permanent waterproof marker pens, or on
an adhesive label where necessary. Never solely label the cap of a
container: the body of the container should always be the primary label
(Smith et al., 2015).

For very small containers, it might be necessary to place the label on a
small piece of appropriate material (to ensure neither the sample nor
the label are damaged) inside the container.

If working with alcohol for sample preservation, use pencil to label
your samples.

Carefully consider how your samples are going to be stored and
transported to make sure they will not come into contact with any
chemicals or conditions that may damage the labelling.

\hypertarget{recording-information-about-sample-collection-metadata}{%
\section{\texorpdfstring{Recording information about sample collection
(\protect\hyperlink{metadata}{metadata})}{Recording information about sample collection (metadata)}}\label{recording-information-about-sample-collection-metadata}}

Without \protect\hyperlink{metadata}{information about how, where and
when samples are collected, stored, processed and curated}, it is not
possible to interpret results correctly. Recording this information,
known as \protect\hyperlink{metadata}{\emph{metadata}}, is essential and
should be considered as the important as the samples themselves.

Recording sample contents is essential: this information may be required
by permitting authorities and border control when transporting samples
back for analysis.

\hypertarget{references-6}{%
\section{References}\label{references-6}}

Copp, A. J., Kennedy, T. A. and Muehlbauer, J. D. (2014) `Barcodes Are a
Useful Tool for Labeling and Tracking Ecological Samples', Ecological
Society of America, 95(3), pp.~293--300. doi:
\href{https://doi.org/10.1890/0012-9623-95.3.293}{10.1890/0012-9623-95.3.293}.

Smith, P. G., Morrow, R. H. and Ross, D. A. (eds) (2015) Field trials of
health interventions: a toolbox. 3rd edn. Oxford: Oxford University
Press.

\hypertarget{field-guide}{%
\part{Field guide}\label{field-guide}}

\hypertarget{introduction-1}{%
\chapter{Introduction}\label{introduction-1}}

This section provides a guide to good data management for undertaking
fieldwork. It is split into three distinct sections: planning before you
go, whilst in the field, and things to consider upon your return from
the field.

Within in each section, we consider whether you will be collecting data
from an instrument, data by hand or collecting samples and have sets of
key points to think about.

The field guide often refers to the earlier part of this document for
information or guidance on a particular topic, such as data storage
media. We assume you are familiar with these earlier sections and
recommend returning to the relevant sections when you come across these
references to have a full grasp of the context.

\hypertarget{before-you-go}{%
\chapter{Before you go}\label{before-you-go}}

Collecting data in the field requires careful thinking beforehand,
particularly if you are setting up an instrument in a remote environment
that is difficult to reach to troubleshoot, move out of danger or to
recover data from when the data storage is full.

The main points to plan are the same, whether you are collecting
samples, data by hand, using an instrument to gather data in an
automated manner, or a mixture of all of these:

\begin{itemize}
\tightlist
\item
  carefully plan how you \textbf{organise} your saved data or recording
  of samples so it is well thought-out;
\item
  ensure you have enough \textbf{\protect\hyperlink{storing-data}{data
  storage}} or notebooks to record data;
\item
  ensure you can \textbf{\protect\hyperlink{backing-up-data}{back up}}
  your data in the field;
\item
  plan which \textbf{metadata} (information about data) you will record
  about the data or sample collection;
\item
  understand what information you need to \textbf{document} your data or
  sample collection thoroughly.
\end{itemize}

The sections below will lead you through these points in more detail.
Follow the links to the relevant sections for more detailed information
and context.

\hypertarget{on-site}{%
\section{On site}\label{on-site}}

In addition to these points, it is important to find out about what
\textbf{is} and what \textbf{is not} available to you where you might be
undertaking your field work. This will affect how you save, backup and
possibly access data.

\hypertarget{network}{%
\subsection{Network}\label{network}}

Having a connection to the local network can be useful for having
automated backups (if there is also storage), collaborate more easily
with others, view your data without having to access your instrument
(very useful in bad weather) and depending on the organisation, might
mean your data can be backed-up directly to your institution after the
fieldwork.

\textbf{Questions to ask}:

\begin{itemize}
\tightlist
\item
  Do you have access to a local network? If so, is this from a computer
  based at the site, or can you connect your computer?
\item
  Can your instrument be connected to the network? Is there network
  where the instrument will be based, or do you need to think about
  taking longer cables? Is it too far away from a network point?
\item
  What is the speed of the network?
\item
  Are network cables provided?
\item
  Is there someone who can assist in connecting your computer or
  instrumentation to the network, or do you need to know how to do this?
\end{itemize}

\hypertarget{data-storage}{%
\subsection{Data storage}\label{data-storage}}

Network-attached storage may be provided and therefore would offer
another means of managed data backup. It is likely to be more secure
than a portable hard disk if the infrastructure is well-managed.

\textbf{Questions to ask}:

\begin{itemize}
\tightlist
\item
  Are you able to backup data to the network-attached storage? If so,
  what capacity would be available to you?
\item
  Do you have read access to look at your data?
\item
  Is access to read data available any time? Is this possible through
  your computer or is it through a computer based on the site?
\item
  Is there someone who can assist in setting up your access to the data
  storage, or do you need to know how to do this?
\end{itemize}

\hypertarget{power-and-electricity-supply}{%
\subsection{Power and electricity
supply}\label{power-and-electricity-supply}}

In addition to the power supply needed for your instrumentation, you may
need to consider what is available to power extra data storage and
associated UPSs (uninterrupted power supply; keeps instrumentation
running using batter power for a short period if the main supply fails).

\hypertarget{support}{%
\subsection{Support}\label{support}}

Some camps or bases might be able to provide some support or have strict
rules about connecting to local networks. It really helps to understand
these before you arrive.

\hypertarget{preparing-for-data-collection-from-an-instrument}{%
\section{Preparing for data collection from an
instrument}\label{preparing-for-data-collection-from-an-instrument}}

Whilst planning, think carefully about the following questions:

\begin{itemize}
\tightlist
\item
  how much data are you planning to collect?
\item
  do you have enough data storage for the planned data collection, plus
  extra for unforeseen circumstances?
\item
  how will you organise the data files?
\item
  how will the data be backed-up in the field?
\item
  will you be able to access the data during collection?
\item
  what do you need to know about how your data were collected?
\end{itemize}

\hypertarget{how-much-data-are-you-planning-to-collect}{%
\subsection{How much data are you planning to
collect?}\label{how-much-data-are-you-planning-to-collect}}

It is important to have a good understanding of your instrument,
associated software and how the files are saved.

\textbf{Which data do I need to save?}

Some software automatically writes ``false'' data from variables which
is not actually being collected, and therefore should not be recorded in
the data file. This easily causes confusion for yourself and others in
the future, wondering what the data are. Make sure you know how to
``select'' which variables are saved into the data files.

If data storage is limited, maybe you need to consider which periods of
time are more crucial (i.e.~data collection only at night, or for five
minutes every hour, one-minute resolution instead of one-second). All of
this should be considered with your experiment in mind to ensure it is
not compromised. Don't forget to take into account local sunrise and
sunset times if this is an important factor, particularly if the
instrument is going to be installed on a moving platform such as a ship.

\textbf{Do I really understand how big each data file will be?}

File sizes vary depending on the configuration of the instrument, the
number of variables recorded, and importantly, the data themselves
(i.e.~more background noise can produce higher figures and therefore
more bytes). Ensure you configure the instrument for your needs in case
settings have changed since you last used it. Have a proper test-run
before going into the field with the same settings you will use in the
field and compare test data with other similar data wherever possible.

\textbf{How do I calculate how much data I am going to collect?}

Once you know how much a data file produces and how many data files you
will produce (one per hour, one per day?), think about how many MB, GB
or TB per day that you will generate, then multiply that calculation by
the number of days you expect to be collecting data for.

Always round estimates up. It is better to overestimate.

\textbf{Do I have enough data storage for the planned data collection,
plus extra for unforeseen circumstances?}

You will likely need to undertake initial set-up tests in the field
before or during deployment which could use up extra storage space. This
is important to do, so budget space accordingly.

In the event of bad weather and not being able to access the instrument
or other unforeseen circumstances such as your field season being
extended, data may be collected over a longer period. Don't miss out on
the opportunity for additional or opportunistic data collection if it
becomes available, just because you don't have big enough data storage!

Primary storage, that is where the raw copy of your data will be saved
initially, should be of a volume that more than covers the data that you
plan to collect. Always ensure you have a buffer of at least 20 \%,
preferably more, and test how the files are stored thoroughly
beforehand. If in doubt, have more storage rather than less.

\textbf{Metadata and documentation}

More information will follow below about planning for recording
\protect\hyperlink{metadata}{metadata} and
\protect\hyperlink{documentation}{documentation} in the field, but we
must point out here, that it is important also to ensure you have enough
data storage for these important aspects of your data collection. This
could include spreadsheets containing notes and supplementary data,
photographs of experiment setups, digitised hand-written notes or
anything else that could be useful. Use of photographs and video could
be particularly large in terms of data storage and backups, so it is
worth bearing in mind.

\hypertarget{data-storage-media}{%
\subsection{Data storage media}\label{data-storage-media}}

Carefully consider the \protect\hyperlink{storing-data}{hardware} on
which your data will be stored, ensuring that it can withstand the
conditions where it will be. If you don't know what to expect, find out
from others that have been on a similar field trip or have previously
been to the same location.

At permanent field camps, bases or on ships, familiarise yourself with
what is and just as importantly, what is not available to you for data
storage. Consider:

\begin{itemize}
\tightlist
\item
  network (your access, speed, security arrangements and restrictions)
\item
  internet connection (your access and bandwidth)
\item
  data storage - this might or might not be available, but you still
  need some portable media or a good internet connection to be able to
  take the data home and do backups
\end{itemize}

See the main section about \protect\hyperlink{storing-data}{data storage
media} and the earlier section of this chapter about
\protect\hyperlink{on-site}{what might be available to you on site}, for
more information.

If your instrumentation and data storage are connected to an electrical
supply, maybe you need to consider an external power supply such as a
UPS, to keep them running in case of power loss. In this case, find out
details of the
\protect\hyperlink{power-and-electricity-supply}{electricity supply}
where your instrumentation and data storage will be located, to ensure
compatibility.

\hypertarget{organising-data-files}{%
\subsection{Organising data files}\label{organising-data-files}}

Think carefully about the
\protect\hyperlink{directory-structure}{directory structure} and
\protect\hyperlink{file-naming}{filenames} you use, particularly if you
are collecting data automatically. Refer to the relevant sections on
this for more information and where possible, set up the instrument
accordingly beforehand.

\hypertarget{backing-up-data-in-the-field}{%
\subsection{Backing up data in the
field}\label{backing-up-data-in-the-field}}

Always ensure you can \protect\hyperlink{backing-up-data}{back up} your
data and metadata whilst in the field. Plan carefully to make sure these
backups are automated as far as possible, making it much less of a chore
and harder to make a mistake. Test out each method of backup carefully
before you leave to ensure the method and the hardware (if applicable)
work properly. Don't forget to test recovering backups as well.

Depending on the circumstances, backups could be:

\begin{itemize}
\tightlist
\item
  a number of \protect\hyperlink{portable-media}{portable hard drives},
  held by different members of the team wherever possible;
\item
  if others are coming and going to a field site during the season,
  consider asking a responsible person to carry a copy of the data back
  to your institution. This gives you a copy in a different location and
  means it could also be placed on secure networked storage as an extra
  precaution. It would be particularly useful if weather conditions at
  the field site make it difficult to keep portable media safe and in
  good condition;
\item
  on network-attached storage if accessible;
\item
  by sending files using a mobile or satellite connection from the
  instrument (will depend on situation and cost) to cloud (or other)
  storage.
\end{itemize}

Using on-site options such as network-attached storage or sending files
via the Internet are really a bonus option, so always plan to have a
backup plan in place, in case this doesn't work out.

\hypertarget{accessing-data-in-the-field}{%
\subsection{Accessing data in the
field}\label{accessing-data-in-the-field}}

Being able to access your data in the field is extremely useful and
cannot be underestimated. In particular it allows you to:

\begin{itemize}
\tightlist
\item
  check the instrument is running as expected;
\item
  confirm data files are being saved as expected;
\item
  look for interesting features in the data that might indicate
  problems. Knowing about them in near-real time can be hugely
  advantageous when quality checking and processing data.
\end{itemize}

Setting up quick visualisations of data files saves a lot of time and
can tell you a lot with a quick glance.

If you are staying with your instrumentation, it is very likely you will
be able to set-up the instrument, do some tests and maybe see the data
being collected. Checking the data files periodically means you can spot
obvious issues with the instrument early and ensure data are being saved
as you expected (parameters, file format, frequency of records).

If however you are setting up your instrumentation then expecting to
leave it for a period of time, it is worth considering what access you
(or others) might have to it. Running initial tests whilst still with
the instrument in the field is essential. Once you are sure data are
being collected as you expected, then it is useful to be able to access
data periodically from wherever you are. This could be across a network
(for example if you are on a ship or at a base) or even using mobile or
satellite communications if the instrument is isolated. Even if you are
not able to see \emph{all} of the data being collected, a small daily
file with a subset of the data could be enough for you to check that
everything is going well or flag up issues. Of course this is of more
use if you or someone else is then able to go and fix the problem.

For instruments that are very isolated and there is no possibility of
being able to access them, consider if there is a way to remotely
connect to the instrument. This might offer the possibility to restart
it for example, or change crucial settings. Set this up and test
thoroughly beforehand.

\hypertarget{preparing-for-recording-data-by-hand}{%
\section{Preparing for recording data by
hand}\label{preparing-for-recording-data-by-hand}}

If recording data by hand directly into a notebook, think carefully what
you would like to record. Keep separate notes (documentation and
\protect\hyperlink{metadata}{metadata}) about how your measurements will
be recorded, units and any parameter abbreviations you will use in the
field.

\hypertarget{data-backup-and-digitisation}{%
\subsection{Data backup and
digitisation}\label{data-backup-and-digitisation}}

Scan, photograph or type-up your hand-collected data and notes as soon
as possible in the field as a form of backup but also to make it easier
to check your data collection.

If using a spreadsheet into which your hand-recorded notes will be
copied, prepare the file template in advance and have a test-run of data
collection and digitising of data. Where possible and appropriate, use
drop-down lists of specified terms within your spreadsheet
(\href{https://www.excelefficiency.com/create-drop-down-lists-in-excel/}{example
for Excel}) to keep data entry consistent. This will allow quick data
validation and save a lot of time cleaning data. As when collecting data
from an instrument, early digitisation also offers the opportunity to
produce some quick visualisations or numerical checks of data.

If you will not be able to use a laptop in the field perhaps due to lack
of electricity supply, then consider other methods. This could be taking
a camera and carefully photographing your notes and data. Photographs of
hand-written notes (or scans once you get back to your institution) are
very useful in case you happen to mistranscribe some information and
need to go back to them, as well as being a vital backup.

See the section above about
\protect\hyperlink{how-much-data-are-you-planning-to-collect}{planning
data storage} to ensure you are prepared.

\hypertarget{preparing-for-sample-collection}{%
\section{Preparing for sample
collection}\label{preparing-for-sample-collection}}

The main things to prepare if you are going to be collecting samples
are:

\begin{itemize}
\tightlist
\item
  a carefully planned numbering and labelling scheme;
\item
  a plan of what metadata you want to know about the samples you have
  collected;
\item
  what additional documentation you will maintain;
\item
  how your metadata files are going to be stored and backed up whilst in
  the field;
\item
  check permit requirements: they may insist on certain information
  being kept about the samples and it is likely you will need to report
  what you have collected. Be aware of this before you go to avoid any
  doubt in the field;
\item
  check border crossing and entry requirements for the transport of your
  samples on return: it is likely you will need to present a list of
  samples with information about what they contain. If you are able to
  maintain this record during your fieldwork it will save a lot of time
  at the end of your fieldwork when you are in a rush to pack. However
  it is important to carefully check and understand exactly what you
  need before you go - at a remote field site you might not be able to
  access this information.
\end{itemize}

\hypertarget{metadata}{%
\section{\texorpdfstring{\protect\hyperlink{metadata}{Metadata}}{Metadata}}\label{metadata}}

Prepare carefully how you are going to record
\protect\hyperlink{metadata}{metadata} and see this section of the guide
for detailed documentation about what information should be recorded.

The most important thing is to be able to determine where and when you
collected your data or samples.

\hypertarget{key-points-where}{%
\subsection{Key points: ``where''}\label{key-points-where}}

Make sure you know in advance:

\begin{itemize}
\tightlist
\item
  how you are going to record the location - if your instrument will
  always be in the same place you can use a hand-held GPS to record its
  position. If you are collecting data or samples in several distinct
  locations, a hand-held GPS will be useful to do this. If your platform
  is moving, for example you are on a ship or an aircraft, and you are
  constantly collecting data, you should have a device constantly
  recording the location of the platform.
\item
  understand how accurate your recording of the location should be to be
  meaningful for your experiment.
\item
  ensure you always have more than one source of location, particularly
  on a moving platform.
\item
  be aware that local conditions such as mountainous or tree-covered
  terrain can affect the accuracy of your location.
\item
  note the source of location data (device name, type, manufacturer,
  version and serial number; see
  \protect\hyperlink{instrumentation-and-computers}{instrumentation}
  section below for details).
\end{itemize}

\hypertarget{key-points-when}{%
\subsection{Key points: ``when''}\label{key-points-when}}

\begin{itemize}
\tightlist
\item
  it is good practice to record all science work in UTC to avoid
  confusion with time zones.
\item
  depending on the accuracy required for your particular experiment,
  ensure your timing device is set accurately. Record if you note the
  time to the nearest day, hour, minute or second.
\item
  note the source of time (device name, type, manufacturer, version and
  serial number).
\item
  if using networked computers or instruments, try to ensure that they
  are all synced to ensure there is no offset.
\item
  if working as part of a larger project, particularly where data and
  sample collection is simultaneous among teams, ensure you are all
  working from the same time source.
\item
  properly understand if there is any offset between the instruments /
  devices recording location, time and your data itself. If on a moving
  platform, it is likely that you will need to match up data points to
  the location using time.
\end{itemize}

\hypertarget{instrumentation-and-computers}{%
\subsection{Instrumentation and
computers}\label{instrumentation-and-computers}}

Keep a detailed record of instrumentation that you use for primary and
secondary datasets, sample collection and saving / backing up data. If
at all possible, record this information in advance before going to the
field and remember to take it with you.

For instrumentation, record:

\begin{itemize}
\tightlist
\item
  name
\item
  type
\item
  manufacturer
\item
  version / model
\item
  serial number
\end{itemize}

Good documentation of instrumentation is important, particularly if
there are any issues. Keep the details to hand in the field and with
anyone who is able to offer support back at your institution: this makes
it much easier to contact the manufacturer for support whilst in the
field, or afterwards when trying to solve problems.

During an expedition, it is possible that part of the instrumentation
might change: for instance, a new sensor could be added, or swapped if
one fails. Always ensure you record:

\begin{itemize}
\tightlist
\item
  date installed
\item
  date removed
\item
  details of problems (if there were any)
\item
  location of installation on the platform or on parent instrument
  (e.g.~side of ship, height on mast, which part of CTD rosette)
\item
  which data files each instrument corresponds to
\end{itemize}

For computers and software, record:

\begin{itemize}
\tightlist
\item
  operating system name and version
\item
  software name and version
\item
  any particular set-up
\end{itemize}

Prepare file templates (spreadsheets are ideal) for collecting this
information before you go in the field so nothing is forgotten.

\hypertarget{documentation}{%
\section{Documentation}\label{documentation}}

Much of the specific \protect\hyperlink{metadata}{metadata} to be
recorded has been described in detail above. Carefully prepare template
documents that you can take and complete whilst in the field.

For further documentation this can be a simple
\protect\hyperlink{readmetxt}{readme.txt} files with headers to remind
you which information you need to record.

Notes, diagrams, photographs and any other forms of documentation should
be backed up alongside your data.

\hypertarget{travel-and-customs}{%
\section{Travel and customs}\label{travel-and-customs}}

Plan how you will get your data storage (and related) hardware to and
from the field location with good time. Carefully check cargo
restrictions to ensure that your hardware meets the requirements of
carriage. Batteries that are in UPSs or other instrumentation are
particularly regulated. Import regulations should also be carefully
checked. Finally, don't forget to consider the regulations of countries
through which you will be transiting, as well as your means of
transport.

Don't forget to plan for the return journey: when returning with
portable media devices holding your carefully-collected data, think
carefully about how you will get them back. Consider encryption of the
device if it holds personal or sensitive data and if someone else is
travelling back to the same location, consider carrying separate copies
of the data.

As previously discussed in the section about
\protect\hyperlink{preparing-for-sample-collection}{preparation for
sample collection}, it is important to understand the requirements of
customs and other permits that are required so you can record such
information (\protect\hyperlink{metadata}{metadata}) whilst in the
field, avoiding a rush at the end of your field work to complete such
documentation.

\hypertarget{in-the-field}{%
\chapter{In the field}\label{in-the-field}}

\hypertarget{data-collection-from-an-instrument}{%
\section{Data collection from an
instrument}\label{data-collection-from-an-instrument}}

Automation, careful setup and testing are key. Much of your setup for
\protect\hyperlink{storing-data}{data storage} should already be planned
and in place before you begin work in the field. Refer to the
\protect\hyperlink{before-you-go}{planning} section for what to think
about and prepare before you arrive in the field.

\hypertarget{initial-setup-and-testing}{%
\subsection{Initial setup and testing}\label{initial-setup-and-testing}}

Retaining data and keeping careful notes
(\protect\hyperlink{metadata}{metadata}) about initial setup and tests
in the field is essential in case you notice any problems later on.

Firstly check that you can read a data file. If files are saved in
proprietary formats, then make sure you have access to the software you
need to be able to check them carefully.

Check that data files contain data for the parameters you expect in the
correct units. Ensure the files are being saved in the
\protect\hyperlink{directory-structure}{directory structure} and with
the \protect\hyperlink{file-naming}{filenames} that you expect. Now is
the time to make any necessary changes.

\hypertarget{periodic-checking-of-data}{%
\subsection{Periodic checking of data}\label{periodic-checking-of-data}}

If you are able to access your data whilst in the field, take advantage
of being able to check files are being saved correctly and that they can
be read. Consult the software documentation for how is best to do this,
to avoid problems which will differ between operating systems and
software. Avoid interfering with the file-writing process, so it is best
to test or check files that belong to a secondary copy if possible.

For data that is being recorded continuously, it would be good practice
to check files at least once a day, or more often if time allows. Events
such as bad weather may mean you need to check the data more regularly.

If you are able to visualise your data in some way, this is a very nice
way to spot anomalies in the data which may in turn help you spot an
instrument problem which you might subsequently be able to fix to get
your data collection back on track.

\hypertarget{backups}{%
\subsection{\texorpdfstring{\protect\hyperlink{backing-up-data}{Backups}}{Backups}}\label{backups}}

\protect\hyperlink{backing-up-data}{Backups} should be automated
wherever possible. If this is not possible, make sure they are done on
at least a daily basis and kept in more than one place. You should have
at least two backups of your data.

Check the integrity of your backups on a regular basis: make sure the
files that are backed up are the same as the originals and that you can
read them correctly.

\hypertarget{handy-tips-4}{%
\subsection{Handy tips}\label{handy-tips-4}}

\begin{itemize}
\tightlist
\item
  Unless you have forgotten (or would like to add) an additional
  parameter to a dataset, do not change what is collected in a dataset
  part way through a field campaign. This adds much unneeded complexity
  and confusion when trying to read the data files during the
  post-processing stages. In particular, do not change the format of
  fields or parameter names.
\end{itemize}

\hypertarget{recording-data-by-hand}{%
\section{Recording data by hand}\label{recording-data-by-hand}}

If collecting data in a notebook by hand, it is good practice to
\protect\hyperlink{data-backup-and-digitisation}{digitise the data} as
soon as possible. Digitised data files, metadata and documentation
should be considered as valuable as any other data file: consider the
\protect\hyperlink{file-naming}{file name},
\protect\hyperlink{directory-structure}{directory structure} and
\protect\hyperlink{backing-up-data}{backups}.

\hypertarget{backing-up-or-digitising-your-data}{%
\subsection{\texorpdfstring{\protect\hyperlink{backing-up-data}{Backing
up} or \protect\hyperlink{data-backup-and-digitisation}{digitising your
data}}{Backing up or digitising your data}}\label{backing-up-or-digitising-your-data}}

Ideally hand-written data should be digitised as soon as possible after
data collection and at least once a day. A simple first backup can be
done by photographing your notes in case the unthinkable happens and you
lose your notebook. Recording hand-written notes in a structured manner
such as in a spreadsheet though, cannot be underestimated and this
should be done as regularly as possible as well (at least daily). This
means you can embellish any shorthand you have used before you forget
what it means, or follow up on anything you needed to query. In
particular in cases where you might be recording species observed for
example, you can follow up on any missing identifications before you
forget key details. See the previous chapter about what to
\protect\hyperlink{preparing-for-data-collection-by-hand}{prepare before
you go} for details of what you can prepare in advance to save time in
the field.

If adding to a data file with new observations, using the date to denote
different versions of the file is helpful e.g.,
\texttt{project\_datatype\_observations\_YYYYMMDD.csv}

Don't forget to \protect\hyperlink{backing-up-data}{backup} your
digitised data as well.

\hypertarget{sample-collection}{%
\section{\texorpdfstring{\protect\hyperlink{collecting-samples}{Sample
collection}}{Sample collection}}\label{sample-collection}}

Record \protect\hyperlink{metadata}{metadata} about
\protect\hyperlink{collecting-samples}{sample collection} as accurately
and as soon as possible. Cross-checking this information is useful to
avoid mistakes.

Ensure you are able to find samples easily from your records,
particularly if you are participating in a long field campaign.
Recording their storage location and box / crate number is also useful
in case you need to find them again, such as on arrival at customs. Also
be sure to note if they are destroyed during any sample analysis that
you undertake whilst in the field: you don't want to be searching for a
sample that no longer exists!

Finally, keep an accurate record of what is contained in each sample
according to what is required for
\protect\hyperlink{travel-and-customs}{permits and border entry
requirements} and where the sample is being sent for analysis following
the field work.

\hypertarget{data-from-sample-analysis}{%
\section{Data from sample analysis}\label{data-from-sample-analysis}}

In some cases, you might collect samples in the field and then process
them whilst still away from your home laboratory. If using
instrumentation to process samples, please refer to the section about
\protect\hyperlink{data-collection-from-an-instrument}{data collection
from an instrument}.

Your field setup will very much dictate how you are able to record data
from sample analysis. If you have to record numerical values from
processing consider if you will be able to have a laptop or computer in
the lab, or if will you be noting the values by hand. If the former,
then treat this as
\protect\hyperlink{data-collection-from-an-instrument}{data collection
from an instrument}, taking good care to regularly save and backup data
files. If the latter, then refer to the section,
\protect\hyperlink{recording-data-by-hand}{recording data by hand}. This
data should be digitised as soon as possible and backed up securely.

If recording further information about the samples such as identifying
different species of plant, then please refer to the section,
\protect\hyperlink{recording-data-by-hand}{recording data by hand}.

\hypertarget{recording-metadata}{%
\section{Recording metadata}\label{recording-metadata}}

Ensure you follow the \protect\hyperlink{metadata}{metadata} guide,
paying particular attention at this point to recording where and when
your data and samples were collected. Metadata is most accurate when
recorded as soon as possible - don't leave it until the end of your
field campaign to make notes.

Remember to digitise this information if you would ordinarily record it
in a notebook and treat these files like you would any other data, using
secure \protect\hyperlink{storing-data}{storage} and regular
\protect\hyperlink{backing-up-data}{backups}.

Take photographs of as many things as you can! Your instrument set-up,
makes, models and serial numbers of equipment equipment, notebooks,
field sites, laboratory setting, your notebooks and anything that could
affect your experiment directly or indirectly. This is hugely valuable
metadata and should be carefully
\protect\hyperlink{backing-up-data}{backed-up}.

\hypertarget{upon-your-return}{%
\chapter{Upon your return}\label{upon-your-return}}

When returning from the field, data management \textbf{priorities} would
be to:

\begin{itemize}
\tightlist
\item
  backup data (data, metadata and documentation) to networked storage:
  ensure the primary copy is read-only;
\item
  ensure safe storage of samples;
\item
  ensure safe storage of hand-written notes.
\end{itemize}

\textbf{Further considerations} will then be to:

\begin{itemize}
\tightlist
\item
  check metadata and documentation and add any extra information whilst
  it is still clear in your mind;
\item
  document clearly where your data are backed up and how they can be
  accessed; ;
\item
  organise a \textbf{copy} of your data in an area where you are able to
  \protect\hyperlink{working-on-your-data}{work on it}.
\end{itemize}

\end{document}

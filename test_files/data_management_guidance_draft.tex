\PassOptionsToPackage{unicode=true}{hyperref} % options for packages loaded elsewhere
\PassOptionsToPackage{hyphens}{url}
\PassOptionsToPackage{dvipsnames,svgnames*,x11names*}{xcolor}
%
\documentclass[12pt,a4paper,oneside]{report}
\usepackage{graphicx}

\usepackage{lmodern}
\usepackage{amssymb,amsmath}
\usepackage{ifxetex,ifluatex}
\usepackage{fixltx2e} % provides \textsubscript
\ifnum 0\ifxetex 1\fi\ifluatex 1\fi=0 % if pdftex
  \usepackage[T1]{fontenc}
  \usepackage[utf8]{inputenc}
  \usepackage{mathptmx} % provides euro and other symbols
\else % if luatex or xelatex
  \usepackage{unicode-math}
  \defaultfontfeatures{Ligatures=TeX,Scale=MatchLowercase}
\fi
% use upquote if available, for straight quotes in verbatim environments
\IfFileExists{upquote.sty}{\usepackage{upquote}}{}
% use microtype if available
\IfFileExists{microtype.sty}{%
\usepackage[]{microtype}
\UseMicrotypeSet[protrusion]{basicmath} % disable protrusion for tt fonts
}{}
\IfFileExists{parskip.sty}{%
\usepackage{parskip}
}{% else
\setlength{\parindent}{0pt}
\setlength{\parskip}{6pt plus 2pt minus 1pt}
}
\usepackage{xcolor}
\usepackage{hyperref}
\hypersetup{
            pdftitle={Data management in the field},
            pdfauthor={Jen Thomas, Swiss Polar Institute},
            colorlinks=true,
            linkcolor=blue,
            citecolor=Blue,
            urlcolor=Blue,
            breaklinks=true}
\urlstyle{same}  % don't use monospace font for urls
\usepackage[top=25mm,left=20mm,bottom=20mm,right=20mm,heightrounded]{geometry}
\setlength{\emergencystretch}{3em}  % prevent overfull lines
\providecommand{\tightlist}{%
  \setlength{\itemsep}{0pt}\setlength{\parskip}{0pt}}
\setcounter{secnumdepth}{0}

% set default figure placement to htbp
\makeatletter
\def\fps@figure{htbp}
\makeatother


\title{Data management in the field}
\author{Jen Thomas, Swiss Polar Institute}
\date{September 2020}
\usepackage{showframe}[geometry]
% test using this new package for the formatting
%\usepackage{scrlayer-scrpage}
%\ohead{}% clear the outer head
%\ihead{\includegraphics[scale=0.35]{spi_logo_hd.png}}
%\addtokomafont{pagehead}{\sffamily}
%\addtokomafont{pagefoot}{\small}% Making the foot extra tiny to demonstrate
%\addtokomafont{pagenumber}{\small\bfseries\color{black}}% that the page number can be controlled on its own.
%\cfoot*{\pagemark}% the pagenumber in the outer part of the foot, also on plain pages
%\ifoot*{\date}% Name and title beneath each other in the inner part of the foot
%\setlength{\footheight}{24.0pt}

%
\usepackage{fancyhdr} % other pages
\pagestyle{fancy}
\fancyhead[LE,LO]{\includegraphics[scale=0.35]{spi_logo_hd.png}}
\fancyfoot[CE,CO]{\fontsize{11}{11}\selectfont \thepage}
\fancyfoot[L]{\fontsize{10}{11}\selectfont \date{September 2020}}
\setlength\textheight{660pt}
  \setlength{\headsep}{15pt}
  \setlength{\headheight}{12pt} % set height of header so it is not affected by header on title page

\renewcommand{\chaptermark}[1]{%
\markboth{\MakeUppercase{%
\chaptername}\ \thechapter.%
\ #1}{}}

\fancypagestyle{plain}{%  first page of chapters
  \fancyhead[L]{\includegraphics[scale=0.35]{spi_logo_hd.png}} % add SPI logo
  \fancyfoot[C]{\fontsize{11}{11}\selectfont \thepage} % add centred page number
  \fancyfoot[L]{\fontsize{10}{11}\selectfont \date{September 2020}}
  \setlength\textheight{660pt}
  \setlength{\headsep}{20pt}
  \setlength{\headheight}{21pt} % set height of header so it is not affected by header on title page
}

\fancypagestyle{titlepagestyle}
{
  \fancyhead[L]{\includegraphics[scale=0.6]{spi_logo_hd.png}\\ \hrulefill \\ EPFL ENT SPI \\ GR C2 505 \\ Station 2 \\ CH-1015 Lausanne, Switzerland}
  \fancyhead[C]{Phone: +41 21 693 76 74 \\ +41 21 693 10 74 \\ Web: www.swisspolar.ch}
  \fancyfoot[C]{\includegraphics[scale=0.25]{spi_funders_logos_png.png}}
  \renewcommand{\headrulewidth}{0pt} % do not display header line
}

\usepackage{xpatch}
\xapptocmd{\titlepage}{\thispagestyle{titlepagestyle}}{}{} % apply titlepagestyle

\begin{document}
\maketitle

{
\hypersetup{linkcolor=}
\setcounter{tocdepth}{1}
\tableofcontents
}
\hypertarget{introduction}{%
\chapter{Introduction}\label{introduction}}

\hypertarget{benefits-of-good-data-management}{%
\section{Benefits of good data
management}\label{benefits-of-good-data-management}}

Setting time aside for carefully managing data and samples from the
beginning of a project, will save a lot of time in the long run.
Investment of time in good preparation and management of data files,
software, documentation and associated files is an investment for the
future. Detailed documentation, organised files and well-kept backups
will make life a lot easier for your future self and others.

\hypertarget{what-this-guide-covers}{%
\section{What this guide covers}\label{what-this-guide-covers}}

In this guide we cover good research data management practice with a
focus on field work, although the general guidelines are applicable in
any circumstances. The first part of the guide is general and can be
applied throughout all stages of the research life cycle.

The second part is a guide for data management in the field and is
broken into three sections: planning data management for fieldwork, some
key points to remember whilst in the field and a brief set of
considerations for returning to your institution. We regularly refer
back to the key sections of the first part of the document and recommend
that the reader is familiar with this before reading the field guide.

\hypertarget{data-management---good-practice}{%
\part{Data management - good
practice}\label{data-management---good-practice}}

\hypertarget{storing-data}{%
\chapter{Storing data}\label{storing-data}}

Storing your data in an organised and secure manner, will save you a lot
of time and hassle in the long run. It is worth investing time and
effort in ensuring a coherent
\protect\hyperlink{directory-structure}{directory structure},
understandable \protect\hyperlink{file-naming}{file names}, secure data
storage and carefully thought-out
\protect\hyperlink{data-backup}{backups}.

In this section we focus on where to store your data. Suggestions are
\emph{not} ordered by preference.

\hypertarget{what-to-consider-when-deciding-where-to-store-your-data}{%
\section{What to consider when deciding where to store your
data}\label{what-to-consider-when-deciding-where-to-store-your-data}}

Thinking carefully about what you need to be able to do with your data
will help you to select where is best to store your data.

\begin{itemize}
\tightlist
\item
  What do you want to do with the data?
\item
  Do other people need access to your data?
\item
  What is the volume of your data?
\item
  How long do you want to use the particular storage for?
\item
  What budget do you have for data storage?
\item
  How much does the data storage cost?
\item
  Are your data sensitive?
\end{itemize}

\hypertarget{types-of-storage-media}{%
\section{Types of storage media}\label{types-of-storage-media}}

Remember that for \protect\hyperlink{data-backup}{backups} of data, it
is important to have them in multiple places and on different types of
media. Different types of media offer solutions to different problems
and therefore you will need to consider using more than one.

\hypertarget{networked-data-storage}{%
\subsection{Networked data storage}\label{networked-data-storage}}

Networked data storage, often known as an institution file server, is
often provided as standard by institutions.

\begin{itemize}
\tightlist
\item
  This is often a handy place to store your data for easy access whilst
  working at your institution.
\item
  Always familiarise yourself with the backup schedule.
\item
  Check if recovery of previous versions of files is necessary (it might
  only be possible if a hard drive is corrupt for example, rather than
  if you overwrite a file by mistake). Find out if you are able to
  recover previous versions of files yourself, or if you need the help
  of an administrator.
\item
  Check access rights: would you have a personal area of data storage,
  or would it be shared amongst your lab or a wider group? Think if you
  need the data to have restricted access, particularly if it is
  sensitive (protected species) or contains personal data. Access should
  also be limited to avoid changing or deleting files accidentally. Make
  sure that primary raw data are read-only for yourself and others.
\item
  It is worth checking if it is easy to allow external collaborators to
  access your data, although this is more often done through other
  means.
\item
  Check what limits there are on the volume of storage: if you are
  collecting large volumes of data, you may need to budget for asking
  for more data storage.
\item
  Consider that you may need to access the data from off-site from time
  to time and find out early on how to do this. Make sure it works,
  there is nothing worse than trying to access data (or other file) that
  you need in a hurry and not have what you need set up properly.
\end{itemize}

\hypertarget{personal-computers}{%
\subsection{Personal computers}\label{personal-computers}}

If using a personal computer to work on data, ensure that you have a
master copy backed up elsewhere, that you work on a \emph{copy} of the
\protect\hyperlink{working-on-your-data}{raw data}, and that any work
you do is backed up on a regular basis.

Failure of the hard disk, theft or damage to a personal computer or
laptop itself could mean loss of all the data and associated files.

\hypertarget{portable-media}{%
\subsection{Portable media}\label{portable-media}}

Portable hard drives are commonly used for
\protect\hyperlink{data-backup}{backing up} data in the field but their
use-case should be carefully considered. Some portable media types are
not useful for long-term storage because they quickly degrade or become
obsolete.

\begin{itemize}
\tightlist
\item
  Buy reputable makes of portable hard drives. If you are going to use a
  lot of them (for example, during a particular field season), by at
  least two different makes to avoid buying a ``bad'' batch.
\item
  Consider buying several medium-sized hard drives rather than one large
  one. If one fails, at least you do not lose everything.
\item
  Take care of hard drives: remember that they are susceptible to
  physical damage and depending on how many times you write to them, may
  only last a few years.
\item
  Pen drives are easy to lose and shouldn't be considered as reliable
  for data backup.
\item
  Consider how long it takes to back up your data and bear this in mind.
  If using an older device, it may have USB-2 only which is considerably
  slower than USB-3.
\item
  If a hard drive is lost, it can be easily read by anyone. The hard
  drive should be encrypted if it holds any personal or sensitive data.
\item
  Regularly check any data that are held on portable media to ensure it
  can still be read. Always have other backups.
\end{itemize}

\hypertarget{cloud-storage}{%
\subsection{Cloud storage}\label{cloud-storage}}

Cloud storage is becoming the norm in many cases, particularly where the
data volume is getting into hundreds of GBs. There are many types of
cloud storage that you can set up yourself or buy as a managed service,
but there are a few considerations to take into account. Many
institutions now also offer this either as part of their standard
storage or as an extra service, particularly for those dealing with
larger volumes of data. Research data management staff in your
institution may also be able to offer recommendations.

\begin{itemize}
\tightlist
\item
  If you are storing personal data, it is essential to understand the
  location of the physical servers and ensure this complies with
  regulations such as GDPR or of your institution.
\item
  Access rights are also important. Consider who might have access to
  the data, if you use read-only or write access rights.
\item
  Check the privacy policy carefully.
\item
  Some cloud storage providers charge not only for the data storage, but
  also for the number of files, copying data to and from the storage, as
  well as listing files (known as objects in cloud storage). It is
  important to think about how this could impact on your costs.
\item
  Some providers provide different levels of storage: consider if you
  want to have immediate access to your data or if you are happy to have
  it in ``cold storage'' where it may take a while to access it - this
  is often a low-cost option and good for long-term backups of data that
  you are not actively working on.
\item
  For managed systems, make sure that backups are done regularly.
\item
  For unmanaged systems, check how you will be able to copy and access
  data. Many systems require the use of command-line tools such as
  \texttt{rclone}.
\item
  Ensure you understand how to give others access to the data.
\item
  Speed of access to data (particularly when more than a few GBs) can be
  limited by your bandwidth - consider how you will access large volumes
  of data carefully.
\item
  Always check that you are able to move your data to another provider
  at a reasonable cost and in a reasonable time manner. This could
  happen if there are changes to the services that are provided, they go
  out of service or they no longer meet the requirements you need. Some
  providers may use proprietary formats which might make moving the data
  very hard.
\item
  Ensure the subscription is always maintained (if paid for) otherwise
  you may find your data are deleted.
\end{itemize}

Wikipedia has a very handy
\href{https://en.wikipedia.org/wiki/Comparison_of_online_backup_services}{comparison
of online backup options}. These would normally be used for medium to
long-term storage or backups.

Platforms such as \href{https://zenodo.org}{Zenodo} are provided for
publication of data and other digital resources, but when datasets have
been completed (either in a raw or finalised state) it can be worth
thinking about this option.

Object storage has various idiosyncrasies in terms of differences to
file systems that are useful to be aware of. For example, files are
known as ``objects'', S3 prefixes are not directories (Chan, 2020a) and
S3 keys are not file paths (Chan, 2020b).

\hypertarget{file-organisation}{%
\section{File organisation}\label{file-organisation}}

It is important to strike a balance between the number of files you
produce and their size. Consider the size of data files and the number
of files you store in a directory.

\hypertarget{file-size}{%
\subsection{File size}\label{file-size}}

Consider file size carefully so they are easy to work with for yourself,
future users and applications. Many small files will take longer to copy
and be harder to work with than a single file of the same total size.
Equally, avoid creating files that are more than 1 GB because in some
cases they can be difficult to read into memory. Copying lots of smaller
files to \protect\hyperlink{cloud-storage}{cloud storage} can also
increase the cost which maybe a factor.

\hypertarget{number-of-files-in-a-directory}{%
\subsection{Number of files in a
directory}\label{number-of-files-in-a-directory}}

TODO

\hypertarget{references}{%
\section{References}\label{references}}

Chan, A. (2020). alexwlchan. \emph{S3 keys are not file paths}.
Retrieved from
\url{https://alexwlchan.net/2020/08/s3-keys-are-not-file-paths/}
{[}Accessed on 15 September 2020{]}.

Chan, A. (2020). alexwlchan. \emph{S3 prefixes are not directories}.
Retrieved from
\url{https://alexwlchan.net/2020/08/s3-prefixes-are-not-directories/}
{[}Accessed on 15 September 2020{]}.

Craig-Wood. N. Rclone. (2014-2020). \url{https://rclone.org}

The University of Edinburgh. Storage \& security. \emph{MANTRA Research
Data Management Training}. Retrieved from
\url{https://mantra.edina.ac.uk/storageandsecurity} {[}Accessed on 28
July 2020{]}.

\end{document}
